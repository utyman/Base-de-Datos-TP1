%%This is a very basic article template.
%%There is just one section and two subsections.
\documentclass{article}
\usepackage{ulem}

\def\dotuline{\bgroup 
  \ifdim\ULdepth=\maxdimen  % Set depth based on font, if not set already
   \settodepth\ULdepth{(j}\advance\ULdepth.4pt\fi
  \markoverwith{\begingroup
  \advance\ULdepth0.08ex
  \lower\ULdepth\hbox{\kern.15em .\kern.1em}%
  \endgroup}\ULon}

\def\dashuline{\bgroup 
  \ifdim\ULdepth=\maxdimen  % Set depth based on font, if not set already
   \settodepth\ULdepth{(j}\advance\ULdepth.4pt\fi
  \markoverwith{\kern.15em
  \vtop{\kern\ULdepth \hrule width .3em}%
  \kern.15em}\ULon}

\begin{document}

\section{Esquemas}

\textbf{Veh\'iculo}(\underline{PATENTE}, modelo, marca, antiguedad, situacion)
\\
PK = CK = \{PATENTE\}
\\
FK = \{IDESTADO\}
\\
\\
\textbf{Veh\'iculoEnUso}(\dashuline{\underline{PATENTE}})
\\
PK = CK = FK = \{PATENTE\}
\\
\\
\textbf{Veh\'iculoEnReparacion}(\dashuline{\underline{PATENTE}},
fechaIngresoTaller)
\\
PK = CK = FK = \{PATENTE\}
\\
\\
\textbf{Estado}(\underline{ID}, descripcion)
\\
PK = CK = \{ID\}
\\
\\
Establecemos las siguientes \textbf{restricciones adicionales}:
\\
\\
\textit{VehiculoEnUso.PATENTE} debe estar en \textit{Vehiculo.PATENTE}
\\
\\
\textit{VehiculoEnReparacion.PATENTE} debe estar en \textit{Vehiculo.PATENTE}
\\
\\
\textit{Vehiculo.PATENTE} debe estar en \textit{VehiculoEnUso.PATENTE} o
(exclusivo) en \textit{VehiculoEnReparacion.PATENTE}
\\
\\
\textit{Vehiculo.IDESTADO} debe estar en \textit{Estado.ID}
\\
\\
\textit{Estado.ID} puede no estar en \textit{Vehiculo.IDESTADO}
\\
\\
El atributo \textit{situacion} en el esquema \textit{Vehiculo} permite
particionar en forma disjunta el conjunto de veh\'iculos.
\\
\\
\textbf{Recorrido}(\dashuline{\underline{numeroOrigen}},
\dashuline{\underline{calleOrigen}},
\dashuline{\underline{ciudadOrigen}}, \dashuline{\underline{numeroDestino}},
\dashuline{\underline{calleDestino}}, \dashuline{\underline{ciudadDestino}})
\\
PK = CK = \{(numeroOrigen, calleOrigen, ciudadOrigen, numeroDestino,
calleDestino, ciudadDestino)\}
\\
FK = \{numeroOrigen, calleOrigen, numeroDestino, calleDestino\}
\\
\\
\textbf{Direccion}(\underline{numero}, \underline{calle},
\dashuline{\underline{IDCIUDAD}})
\\
PK = CK = \{(numero, calle, IDCIUDAD)\}
\\
FK = \{IDCIUDAD\}
\\
\\
\textbf{Ciudad}(\underline{IDCIUDAD}, nombreCiudad)
\\
PK = CK = \{IDCIUDAD\}
\\
\\
\textbf{Ruta}(\underline{IDRUTA}, \dashuline{\underline{numeroOrigen}},
\dashuline{\underline{calleOrigen}},
\dashuline{\underline{ciudadOrigen}}, \dashuline{\underline{numeroDestino}},
\dashuline{\underline{calleDestino}}, 
\dashuline{\underline{ciudadDestino}}, descripcion, km, tiempoEstimado,
nroPeajes)
\\
PK = CK = \{(IDRUTA, (numeroOrigen, calleOrigen, ciudadOrigen, numeroDestino,
calleDestino, ciudadDestino))\}
\\
\\
\textbf{PeriodoA\~no}(\underline{IDPERIODOA\~NO}, descripcion)
\\
PK = CK = \{IDPERIODOA\~NO\}
\\
\\
\textbf{CondicionClimatica}(\underline{IDCONDICIONCLIMATICA}, descripcion)
\\
PK = CK = \{IDCONDICIONCLIMATICA\}
\\
\\
\textbf{Situacion}(\underline{IDRUTA}, \dashuline{\underline{numeroOrigen}},
\dashuline{\underline{calleOrigen}},
\dashuline{\underline{ciudadOrigen}}, \dashuline{\underline{numeroDestino}},
\dashuline{\underline{calleDestino}}, 
\dashuline{\underline{ciudadDestino}}, \dashuline{\underline{IDPERIODOA\~NO}},
\dashuline{\underline{IDCONDICIONCLIMATICA}})
\\
PK = CK = \{( (IDRUTA, (numeroOrigen, calleOrigen, ciudadOrigen, numeroDestino,
calleDestino, ciudadDestino)), IDPERIODOA\~NO, IDCONDICIONCLIMATICA )\}
\\
\\
\textbf{Viaje}(\underline{IDVIAJE}, contingencias, horaPartida, fechaPartida,
\dashuline{\underline{IDRUTA}})
\\
PK = CK = \{IDVIAJE\}
\\
\\
\textbf{ViajeRealizado}(\underline{IDVIAJE}, fechaLlegada, horaLlegada)
\\
PK = CK = \{IDVIAJE\}
\\
\\
\textbf{Chofer}(\underline{DNI}, telefono, fechaNacimiento, fechaObtencion,
fechaRenovacion, domicilio, nombre, nroLicencia, observaciones)
\\
CK=\{DNI, nroLicencia\}
PK = \{DNI\}
\\
\\
\textbf{AsignacionChofer}(\dashuline{\underline{DNI}},
\dashuline{\underline{IDVIAJE}})
\\
PK = CK=\{(DNI, IDVIAJE)\}
\\
FK = \{DNI, IDVIAJE\}
\\
\\
\textbf{Test}(\underline{IDTEST}, \dashuline{\underline{DNI}},
\dashuline{\underline{IDVIAJE}},
resultado, fechaTest, \dashuline{IDTIPOTEST})
\\
PK = CK =\{(DNI, IDVIAJE, IDTEST)\}
\\
FK = \{DNI, IDVIAJE, IDTEST\}
\\
\\
\textbf{TipoTest}(\underline{IDTIPOTEST}, descripcion)
\\
PK = CK =\{IDTIPOTEST\}


\end{document}

