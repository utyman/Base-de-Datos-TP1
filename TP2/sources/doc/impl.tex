\section{Implementaci\'on}

\subsection{Estructura del c\'odigo}
La organizaci\'on del c\'odigo fuente utilizado para este trabajo se basa en Maven. El
c\'odigo fuente entregado por la c\'atedra ya estaba empaquetado para Maven y ven\'ia
provisto de su \textit{pom.xml} con su correspondiente nombre de artefacto y versionado.
Lo desarrollado por el grupo tiene como dependencia de Maven a este proyecto, entonces 
para poder compilar el c\'odigo fuente se necesita tener esta librer\'ia instalada.\\

El c\'odigo fuente desarrollado por el grupo se divide en dos proyectos de Maven.
\begin{itemize}
\item \textbf{UBADB-New}: En este proyecto se implementaron las estrategias MRU y LRU 
que fueron solicitadas por la c\'atedra. Tambi\'en se implementaron otras estrategias 
propuestas por el grupo. Se hizo enf\'asis en la documentaci\'on de este proyecto ya
que contiene estrategias que pueden ser re-utlizables en el futuro.
\item \textbf{UBADB-Benchmark}: En este proyecto se implementaron los procesos para
medir la eficiencia de las estrategias mencionadas. Este proyecto depende del anterior.
\end{itemize}
  
   
\subsection{Estrategias nuevas}
La \'unica estrategia existente en el c\'odigo provisto por la c\'atedra correpond\'ia a la estrategia FIFO 
(el primero en entrar es el primero en salir). Las estrategias implementadas por el grupo tomaron como base esta
implementaci\'on de dos clases: una para la estrategia y otra para agregarle informaci\'on al \textit{BufferFrame}.
\begin{itemize}
\item \textbf{MRU}: esta estrategia registra para cada frame en memoria la fecha en la que fue referenciado por \'ultima vez. Para seleccionar la v\'ictima itera a trav\'es de todas las p\'aginas en memoria y selecciona la que fue referenciada m\'as recientemente.
\item \textbf{LRU}: al igual que MRU, lleva un registro de la \'ultima vez que fue accedida cada p\'agina y escoge como v\'ictima a la que fue referenciada menos recientemente.
\item \textbf{Count}: esta estrategia lleva una contabilidad de los pedidos a cada p\'agina y para elegir la v\'ictima, escoge 
la p\'agina menos pedida hasta el momento. Por esto se la conce como ``No usada frecuentemente'' o NFU.
\item \textbf{Best}: esta estrategia implementa el algoritmo de reemplazo de p\'agina \'optimo de \textit{Belady}. utiliza el conocimiento de la secuencia completa de pedidos antes de ser ejecutada, por lo tanto no es \'util desde el punto de vista pr\'actico
pero s\'i para comparar con el resto de las estrategias. Selecciona como v\'ictima la p\'agina con mayor distancia a su pr\'oxima referencia, o bien a una de aquellas p\'aginas que no van a ser m\'as referenciadas en la traza.
\item \textbf{Random}: la estrategia RANDOM devuelve como v\'ictima un frame al azar. Ser\'a usada para 
comparar eficiencia con otras estrategias.
\end{itemize}
Para seguir manteniendo el orden de c\'odigo estas estrategias se implementaron cada un en un sub-paquete de java propio 
bajo \textit{ubadb.components.bufferManager.bufferPool.replacementStrategies}.

\subsection{Tests de unidad}

Para probar la correctitud de cada una de las estrategias utilizadas, se realizaron una serie de tests de unidad. Estas pruebas se encuentran en los paquetes

\textit{ubadb.components.bufferManager.bufferPool.replacementStrategies.beststrategy.$<$estrategia$>$}

de acuerdo al formato dado por la c\'atedra para el ejemplo de FIFO.


\subsection{Mediciones}
Para comparar las distintas estrategias para el \textit{MemoryBuffer} se consider\'o medir
la tasa de aciertos (\textit{hit rate}) de cada una de ellas ante las mismas secuencias 
de pedidos y liberaciones de p\'aginas (trazas). De cada \textit{traza} se entiende la tasa
de aciertos como el porcentaje de \'paginas pedidas donde no se tuvo que acceder a disco
porque ya estaba cargada en el \textit{MemoryBuffer}. Para poder calcular este resultado
se midieron la cantidad de pedidos de paginas (\textit{requests}) y la cantidad de veces
que se tuvo que leer una p\'agina en el disco (\textit{fails}). Con estas dos magintudes
se define la \textit{hit rate} como: \\
\begin{center}
$hit rate = \frac{requests-fails}{requests}$  
\end{center}
La \textit{hit rate} se puede considerar como un porcentaje multiplic\'andolo por $100$.

\subsubsection{Trazas utilizadas}
Las trazas se generan usando la funcionalidad de la clase TraceUtil en el paquete \textit{ubadb.bench.util} del proyecto BENCH. 
Las secuencias de pedidos y liberaciones utilizadas para comparar las mediciones fueron
\begin{itemize}
\item \textit{File Scan}: genera una traza que simula la lectura secuencial de p\'aginas contiguas. Utiliza el m\'etodo \textit{createTracesFileScan} de la clase TraceUtil.
\item \textit{Index Scan Clustered}: genera una traza que simula la lectura de la estructura que funciona como \'indice y luego de la secuencia de p\'aginas ordenadas. Utiliza el m\'etodo \textit{createTraceScanClustered} de la clase TraceUtil.
\item \textit{Index Scan Unclustered}: genera una traza que simula la lectura de la estructura que funciona como \'indice y luego de la secuencia de p\'aginas desordenadas en el archivo. Utiliza el m\'etodo \textit{createTraceScanUnclustered} de la clase TraceUtil.
\item \textit{BNLJ}: genera una traza que simula un BNLJ. Utiliza el m\'etodo \textit{createTracesBNLJ} de la clase TraceUtil.
\end{itemize}

A partir de estos tipos \textit{puros}, se generaron tambi\'en trazas compuestas por una secuencia aleatoria de una cierta cantidad de ellos. Esta estrategia de generaci\'on se encuentra implementada en el m\'etodo \textit{obtainTrace} de TraceUtil.
