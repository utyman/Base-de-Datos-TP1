\section{Implementaci\'on}

\subsection{Estructura del codigo}
La organizaci\'on del c\'odigo fuente utilizado para este trabajo se basa en Maven. El
c\'odigo fuente entregado por la c\'atedra ya estaba empaquetado para Maven y ven\'ia
provisto de su \textit{pom.xml} con su correspondiente nombre de artefacto y versionado.
Los desarrollado por del grupo tiene como dependencia de Maven a este proyecto, entonces 
para poder compilar el c\'odigo fuente se necesita tener esta librer\'ia instalada.\\
El c\'odigo fuente desarrollado por el grupo se divide en dos proyectos de Maven.
\begin{itemize}
\item \textbf{UBADB-New}: En este proyecto se implementaron las estrategias MRU y LRU 
que fueron solicitadas por la c\'atedra. Tambi\'en se implementaron otras estrategias 
propuestas por el grupo. Se hizo enf\'asis en la documentaci\'on de este proyecto ya
que contiene estrategias que pueden ser re-utlizables en el futuro.
\item \textbf{UBADB-Benchmark}: En este proyecto se implementaron los procesos para
medir la eficiencia de las estrategias mencionadas. Este proyecto depende del anterior.
\end{itemize}
  
   
\subsection{Estrategias nuevas}
La \'unica estrategia existente en el c\'odigo provisto por la c\'atedra correpondia a la estrategia FIFO 
(el primero en entrar es el primero en salir). Las estrategias implementadas por el grupo tomaron como base esta
implementaci\'on de dos clases: una para la estrategia y otra para agregarle informaci\'on al \textit{BufferFrame}.
\begin{itemize}
\item \textbf{MRU}: ... 
\item \textbf{LRU}: ...
\item \textbf{Count}: Esta estrategia lleva una contabilidad de los pedidos a c\'ada p\'agina y para elegir la v\'ictima, escoge 
la p\'agina menos pedida hasta el momento. Por esto se la conce como ``No usada frecuentemente'' o NFU.
\item \textbf{Best}: Esta estrategia implementa el algoritmo de reemplazo de p\'agina \'optimo de \textit{Belady}. Esta estrategia 
utiliza el conociemiento de la sequencia completa de pedidos antes de ser ejecutada, por lo tanto no es \'util desde el punto de vista pr\'actico
pero si para comparar con el resto de las estrategias.  
\item \textbf{Random}: La estrategia random devuelve como v\'ictima un frame al azar, ser\'a usada para 
comprar eficiencia con otras estrategias.
\end{itemize}
Para seguir manteniendo el orden de c\'odigo estas estrategias se implementaron cada un en un sub-paquete de java propio 
bajo \textit{ubadb.components.bufferManager.bufferPool.replacementStrategies}.

\subsection{Tests de unidad}
aca hay que hablar de junit y los junits que hicimos para testear, que testean cada cosa.
(hernan?)

\subsection{Mediciones}
Para comparar las distintas estrategias para el \textit{MemoryBuffer} se consider\'o medir
la tasa de aciertos (\textit{hit rate}) de cada una de ellas ante las mismas secuencias 
de pedidos y liberaciones de p\'aginas (trazas). De cada \textit{traza} se entiende la tasa
de aciertos como el porcentaje de \'paginas pedidas donde no se tuvo que acceder a disco
porque ya estaba cargada en el \textit{MemoryBuffer}. Para poder calcular este resultado
se midieron la cantidad de pedidos de paginas (\textit{requests}) y la cantidad de veces
que se tuvo que leer una p\'agina en el disco (\textit{fails}). Con estas dos magintudes
se define la \textit{hit rate} como: \\
\begin{center}
$hit rate = \frac{requests-fails}{requests}$  
\end{center}
La \textit{hit rate} se puede considerar como un porcentaje multiplic\'andolo por $100$.

\subsubsection{Trazas utilizadas}
Las sequencias de pedidos y liberaciones utilizadas para comparar las mediciones fueron
\begin{itemize}
\item \textit{File Scan}: bla bla file scan
\item \textit{Index Scan Clustered}: bla bla
\item \textit{Index Scan Unclustered}: bla bla
\item \textit{BNLJ}: bla bla
\end{itemize}
   
