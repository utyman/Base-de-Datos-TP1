\section{Implementaci\'on}

\subsection{Motor de bases de datos}
El motor de base de datos utilizado para implementar la soluci\'on es MySQL. Fue elegido por ser el m\'as 
conocido por el equipo de trabajo y por ser un software de bases de datos libre con una importante comunidad de usuarios. \\
Lo que sigue en esta secci\'on es el c\'odigo fuente para la creaci\'on del modelo. Estos c\'odigos fuentes tambien se pueden
encontrar en el directorio ``impl/'' de la entrega.

\subsection{Choferes}

\definecolor{sqlCode}{rgb}{1,0.0,0.5}


\newenvironment{sql}
{\begin{flushleft}\begin{alltt}\color{sqlCode}}
{\end{alltt}\end{flushleft}}


\begin{sql}

CREATE TABLE  `choferes` (
  `dni` varchar(10) NOT NULL,
  `telefono` varchar(20) NOT NULL,
  `fecha_nacimiento` date NOT NULL,
  `domicilio` varchar(30) NOT NULL,
  `nombre` varchar(100) NOT NULL,
  `apellido` varchar(100) NOT NULL,
  PRIMARY KEY (`dni`)
) ENGINE=InnoDB DEFAULT CHARSET=latin1;

CREATE TABLE  `licencias` (
  `nro_licencia` int(11) NOT NULL,
  `fecha_otorgamiento` date NOT NULL,
  `tipo` varchar(100) NOT NULL,
  `observaciones` varchar(100) NOT NULL,
  `fecha_renovacion` datetime NOT NULL,
  `dni_chofer` varchar(10) NOT NULL,
  `fecha_vencimiento` datetime NOT NULL,
  PRIMARY KEY (`nro_licencia`,`fecha_otorgamiento`,`tipo`),
  KEY `dnis_constraint` (`dni_chofer`),
  CONSTRAINT `dnis_constraint` FOREIGN KEY (`dni_chofer`) REFERENCES `choferes` (`dni`)
) ENGINE=InnoDB DEFAULT CHARSET=latin1

\end{sql}

\subsection{Veh\'iculos}

\begin{sql}


CREATE TABLE  `marcas` (
  `id_marca` int(11) NOT NULL,
  `nombre` varchar(100) NOT NULL,
  PRIMARY KEY (`id_marca`)
) ENGINE=InnoDB DEFAULT CHARSET=latin1

CREATE TABLE  `bases`.`modelos` (
  `id_modelo` int(11) NOT NULL,
  `nombre` varchar(30) NOT NULL,
  `id_marca` int(11) NOT NULL,
  PRIMARY KEY (`id_modelo`,`id_marca`) USING BTREE,
  KEY `marca_fk_constraint` (`id_marca`),
  CONSTRAINT `marca_fk_constraint` FOREIGN KEY (`id_marca`) REFERENCES `marcas` (`id_marca`)
) ENGINE=InnoDB DEFAULT CHARSET=latin1;

CREATE TABLE  `vehiculos` (
  `patente` varchar(10) NOT NULL,
  `ano` int(11) NOT NULL,
  `situacion` tinyint(1) NOT NULL,
  `id_modelo` int(11) NOT NULL,
  `id_marca` int(11) NOT NULL,
  `fecha_inicio_servicio` datetime NOT NULL,
  PRIMARY KEY (`patente`),
  KEY `veh_modelo_constraint` (`id_modelo`,`id_marca`),
  CONSTRAINT `veh_modelo_constraint` FOREIGN KEY (`id_modelo`, `id_marca`) 
  REFERENCES `modelos` (`id_modelo`, `id_marca`)
) ENGINE=InnoDB DEFAULT CHARSET=latin1 ROW_FORMAT=DYNAMIC;

CREATE TABLE  `vehiculos_en_reparacion` (
  `patente` varchar(10) NOT NULL,
  `fecha_ingreso_taller` datetime NOT NULL,
  PRIMARY KEY (`patente`),
  CONSTRAINT `patenet_fk_constraint` FOREIGN KEY (`patente`) REFERENCES `vehiculos` (`patente`) 
  ON DELETE NO ACTION ON UPDATE NO ACTION
) ENGINE=InnoDB DEFAULT CHARSET=latin1;


\end{sql}

\subsection{Recorridos y rutas}

\begin{sql}
CREATE TABLE  `ciudades` (
  `id_ciudad` int(11) NOT NULL,
  `nombre_ciudad` varchar(30) NOT NULL,
  PRIMARY KEY (`id_ciudad`)
) ENGINE=InnoDB DEFAULT CHARSET=latin1;

CREATE TABLE  `recorridos` (
  `id_recorrido` int(11) NOT NULL,
  `id_ciudad_origen` int(11) NOT NULL,
  `id_ciudad_destino` int(11) NOT NULL,
  `direccion_origen` varchar(100) NOT NULL,
  `direccion_destino` varchar(100) NOT NULL,
  PRIMARY KEY (`id_recorrido`),
  KEY `ciudad_origen_constraint` (`id_ciudad_origen`),
  KEY `ciudad_destino_constraint` (`id_ciudad_destino`),
  CONSTRAINT `ciudad_destino_constraint` FOREIGN KEY (`id_ciudad_destino`) REFERENCES `ciudades` (`id_ciudad`),
  CONSTRAINT `ciudad_origen_constraint` FOREIGN KEY (`id_ciudad_origen`) REFERENCES `ciudades` (`id_ciudad`) ON DELETE NO ACTION ON UPDATE NO ACTION
) ENGINE=InnoDB DEFAULT CHARSET=latin1

CREATE TABLE  `rutas` (
  `id_ruta` int(11) NOT NULL,
  `id_recorrido` int(11) NOT NULL,
  `longitud` int(11) NOT NULL,
  `tiempo_estimado` int(11) NOT NULL,
  `cantidad_peajes` int(11) NOT NULL,
  `descripcion` varchar(100) NOT NULL,
  PRIMARY KEY (`id_ruta`,`id_recorrido`) USING BTREE,
  KEY `recorrido_constraint` (`id_recorrido`),
  CONSTRAINT `recorrido_constraint` FOREIGN KEY (`id_recorrido`) 
  REFERENCES `recorridos` (`id_recorrido`) ON DELETE NO ACTION ON UPDATE NO ACTION
) ENGINE=InnoDB DEFAULT CHARSET=latin1

CREATE TABLE  `condiciones_climaticas` (
  `id_condicion_climatica` int(11) NOT NULL,
  `descripcion` varchar(100) NOT NULL,
  PRIMARY KEY (`id_condicion_climatica`)
) ENGINE=InnoDB DEFAULT CHARSET=latin1;

CREATE TABLE  `periodos_ano` (
  `id_periodo_ano` int(11) NOT NULL,
  `descripcion` varchar(100) NOT NULL,
  PRIMARY KEY (`id_periodo_ano`) USING BTREE
) ENGINE=InnoDB DEFAULT CHARSET=latin1;

CREATE TABLE  `situaciones_climaticas` (
  `id_ruta` int(11) NOT NULL,
  `id_condicion_climatica` int(11) NOT NULL,
  `id_periodo_ano` int(11) NOT NULL,
  PRIMARY KEY (`id_ruta`,`id_condicion_climatica`,`id_periodo_ano`) USING BTREE,
  KEY `condicion_climatica_constraint` (`id_condicion_climatica`),
  CONSTRAINT `condicion_climatica_constraint`
   FOREIGN KEY (`id_condicion_climatica`) REFERENCES `condiciones_climaticas` (`id_condicion_climatica`),
  CONSTRAINT `id_ruta_constraint` FOREIGN KEY (`id_ruta`) REFERENCES `rutas` (`id_ruta`)
) ENGINE=InnoDB DEFAULT CHARSET=latin1
\end{sql}

\subsection{Viajes y Controles}

\begin{sql}

CREATE TABLE  `viajes` (
  `id_viaje` int(11) NOT NULL,
  `id_vehiculo` varchar(10) NOT NULL,
  `contingencias` varchar(300) NOT NULL,
  `fecha_hora_partida` datetime NOT NULL,
  `id_recorrido` int(11) NOT NULL,
  `fecha_hora_llegada_estimada` datetime NOT NULL,
  PRIMARY KEY (`id_viaje`),
  KEY `new_fk_constraint` (`id_vehiculo`),
  KEY `recorridos_constraint` (`id_recorrido`),
  CONSTRAINT `new_fk_constraint` FOREIGN KEY (`id_vehiculo`) REFERENCES `vehiculos` (`patente`),
  CONSTRAINT `recorridos_constraint` FOREIGN KEY (`id_recorrido`) 
  REFERENCES `recorridos` (`id_recorrido`) ON DELETE NO ACTION ON UPDATE NO ACTION
) ENGINE=InnoDB DEFAULT CHARSET=latin1;

CREATE TABLE  `viajes_realizados` (
  `id_viaje` int(11) NOT NULL,
  `fecha_hora_llegada` datetime NOT NULL,
  `id_ruta` int(11) NOT NULL,
  PRIMARY KEY (`id_viaje`),
  KEY `rutas_constraint` (`id_ruta`),
  CONSTRAINT `rutas_constraint` FOREIGN KEY (`id_ruta`) REFERENCES `rutas` (`id_ruta`),
  CONSTRAINT `viajes_realizados_constraint` FOREIGN KEY (`id_viaje`) 
  REFERENCES `viajes` (`id_viaje`)
) ENGINE=InnoDB DEFAULT CHARSET=latin1;

CREATE TABLE  `test` (
  `id_tipo_test` int(11) NOT NULL,
  `dni` varchar(10) NOT NULL,
  `id_viaje` int(11) NOT NULL,
  `resultado` varchar(100) NOT NULL,
  `fecha_realizacion` datetime NOT NULL,
  PRIMARY KEY (`id_tipo_test`,`dni`,`id_viaje`),
  KEY `viaje_test_constraint` (`id_viaje`),
  KEY `dni_test_constraint` (`dni`),
  CONSTRAINT `dni_test_constraint` FOREIGN KEY (`dni`) 
  REFERENCES `choferes` (`dni`) ON DELETE NO ACTION ON UPDATE NO ACTION,
  CONSTRAINT `tipo_test_constraint` FOREIGN KEY (`id_tipo_test`) 
  REFERENCES `tipo_test` (`id_tipo_test`) ON DELETE NO ACTION ON UPDATE NO ACTION,
  CONSTRAINT `viaje_test_constraint` FOREIGN KEY (`id_viaje`) 
  REFERENCES `viajes` (`id_viaje`) ON DELETE NO ACTION ON UPDATE NO ACTION
) ENGINE=InnoDB DEFAULT CHARSET=latin1;

CREATE TABLE  `tipo_test` (
  `id_tipo_test` int(11) NOT NULL,
  `descripcion` varchar(100) NOT NULL,
  PRIMARY KEY (`id_tipo_test`)
) ENGINE=InnoDB DEFAULT CHARSET=latin1;

CREATE TABLE  `asignaciones_chofer` (
  `dni` varchar(10) NOT NULL,
  `id_viaje` int(11) NOT NULL,
  PRIMARY KEY (`dni`,`id_viaje`),
  KEY `viajes_asignacion_constraint` (`id_viaje`),
  CONSTRAINT `dni_asignacion_constraint` FOREIGN KEY (`dni`) 
  REFERENCES `choferes` (`dni`) ON DELETE NO ACTION ON UPDATE NO ACTION,
  CONSTRAINT `viajes_asignacion_constraint` FOREIGN KEY (`id_viaje`) 
  REFERENCES `viajes` (`id_viaje`) ON DELETE NO ACTION ON UPDATE NO ACTION
) ENGINE=InnoDB DEFAULT CHARSET=latin1;

\end{sql}

\subsection{Stored Procedures}

\subsubsection*{RecorridosTodasRutasUsadas}

El siguiente \textit{stored procedure} obtiene todos los recorridos para los cuales se usaron todas las rutas posibles registradas para ese  recorrido, para viajes realizados el año pasado y recorridos asociados a más de una ruta y corresponde al requerimiento listado como 3.I.
Conceptualmente, se trata de una divisi\'on. Se obtienen todos los recorridos, tales que no existen rutas de ese recorrido para las que no ha habido viajes del \'ultimo a\~no que la usaran. El \'ultimo año se pasa como par\'ametro y se usan funciones auxiliares:

\begin{sql}
CREATE DEFINER=`root`@`localhost` PROCEDURE  
	`RecorridosTodasRutasUsadas`(IN ano integer)
BEGIN
		       SELECT r.id_recorrido
	       	       FROM recorridos r
	       	       WHERE NOT EXISTS
			 (SELECT *
			  FROM rutas rut
			  WHERE rut.id_recorrido = r.id_recorrido AND NOT EXISTS (
				SELECT *
				 FROM viajes_realizados v
				 NATURAL JOIN viajes v1
				 WHERE YEAR(v.fecha_hora_llegada) = ano
				       AND v.id_ruta = rut.id_ruta
				       AND v1.id_recorrido = r.id_recorrido
				) 
			 )			
		      AND ((select cantidadRutasPorRecorrido(r.id_recorrido)) > 1);
		END
		
CREATE DEFINER=`root`@`localhost` FUNCTION  
`cantidadRutasPorRecorrido`(id_recorrido integer) RETURNS int(11)
begin
    RETURN (SELECT count(ruta.id_ruta) FROM rutas ruta WHERE ruta.id_recorrido = id_recorrido);
end
\end{sql}

El siguiente \textit{stored procedure} obtiene el promedio de viajes realizados por veh\'iculo por a\~no y el estado en que \'este se encuentra, corresponde al requerimiento listado como 3.III.

\begin{sql}
CREATE DEFINER=`root`@`localhost` FUNCTION  `bases`.`promedioViajesRealizadosPorAno`
								(patente varchar(10)) RETURNS int(11)
begin
    DECLARE ano_inicio_servicio int;
    DECLARE viajes_realizados int;
    SELECT count(*) into viajes_realizados FROM viajes_realizados 
    		NATURAL JOIN viajes v WHERE  v.id_vehiculo = patente;
    SELECT YEAR(fecha_inicio_servicio)  into ano_inicio_servicio 
    		FROM vehiculos v where v.patente = patente;
    RETURN (viajes_realizados / (YEAR(NOW()) - ano_inicio_servicio + 1)) ;
end

CREATE DEFINER=`root`@`localhost` PROCEDURE  `bases`.`promedioViajesYEstado`()
BEGIN
		SELECT (SELECT promedioViajesRealizadosPorAno(patente)), situacion from vehiculos;
		END
		
		
\end{sql}

El siguiente \textit{stored procedure} obtiene los choferes que han utilizado todos los veh\'iculos de menos de dos a\~nos 
de antiguedad, en viajes del \'ultimo semestre. Corresponde al requerimiento listado como 3.IV.

\begin{sql}
		
CREATE DEFINER=`root`@`localhost` PROCEDURE  
		`ChoferesTodosLosVehiculosUltimoSemestre`()
BEGIN
		       SELECT *
	       	       FROM choferes c
	       	       WHERE NOT EXISTS
			 (SELECT *
			  FROM vehiculos v
			  WHERE (SELECT agregardosanos(v.patente)) > NOW() AND NOT EXISTS (
				SELECT *
				 FROM viajes_realizados v1
				 NATURAL JOIN viajes v2
				 WHERE (SELECT agregarseismeses(v1.fecha_hora_llegada)) > NOW()
				       AND v2.id_vehiculo = v.patente
				 AND EXISTS (
				      SELECT * FROM asignaciones_chofer WHERE
					 dni = c.dni AND id_viaje = v2.id_viaje
				)      
				) 
			 );
		END
		
		CREATE DEFINER=`root`@`localhost` FUNCTION  
		`agregardosanos`(id_vehiculo varchar(10)) RETURNS datetime
BEGIN
		return DATE_ADD((SELECT fecha_inicio_servicio FROM vehiculos where patente = id_vehiculo), 
						INTERVAL 2 YEAR);
		END
		

		CREATE DEFINER=`root`@`localhost` FUNCTION  `bases`.`agregarseismeses`(fecha DATETIME) 
		RETURNS datetime
BEGIN
		return DATE_ADD(fecha, INTERVAL 6 MONTH);
		END
		
		
\end{sql}

