\section{An\'alisis}
En esta secci\'on se analiza el documento de enunciado presentado por la c\'atedra.
De este documento se extraen las caracter\'isticas fundamentales y necesarias que se espera de la soluci\'on. 
La identificaci\'on y organizaci\'on de estos requerimientos, es de gran utilidad para verificar que la soluci\'on planteada
satisface las caract\'eristicas del problema y contiene los entregables solicitados .\\
\subsection{Entregables}
Los entregables solicitados se identifican en la secci\'on ``Consignas de la primera parte'' del enunciado
\begin{enumerate}
\item Modelado
  \begin{enumerate}[label=\Roman{*}]
  \item Modelo de Entidad-Relaci\'on: item a)
  \item Modelo Relacional: item a)
  \item Detalles del modelo y verificaci\'on: items b) y d)
  \end{enumerate}
\item Implementaci\'on
  \begin{enumerate}[label=\Roman{*}]
  \item Dise\~no f\'isico: item c)
  \item Procedimientos y triggers: item e)
  \end{enumerate}
\item Testing
  \begin{enumerate}[label=\Roman{*}]
  \item Datos de prueba: segundo p\'arrafo
  \item Casos de prueba: segundo p\'arrafo
  \end{enumerate}
\end{enumerate}

\subsection{Requerimientos}
Los requerimientos son las caracter\'isticas fundamentales y esperadas que la soluci\'on debe satisfacer y se identifican en
la secci\'on ``Descripci\'on del problema''. 
\subsubsection{Resumen}
El enunciado describe como se relacionan los recursos humanos y materiales de una empresa de transporte y plantea 
la necesidad de un sistema inform\'atico que ayude a administrar dicha empresa. A continuaci\'on se detallan 
los requerimientos sobre este sistema que surgen de la descripci\'on del problema.
\subsubsection{Lista detallada}
\begin{enumerate}
\item Registros principales: el sistema debe ser capaz de registrar los datos que a la empresa le interesa mantener de sus choferes,
  sus veh\'iculos y los recorridos que ofrece. Estos registros son la base para administrar y controlar los viajes y las pol\'itcas de la empresa.
  \begin{enumerate}[label=\Roman{*}]
  \item Registro de choferes
    \begin{itemize}
      \item Datos personales: apellido, nombre, documento, fecha de nacimiento, domicilio y tel\'efono.
      \item Datos de licencia de conducir: n\'umero, tipo, fecha de otorgamiento, fecha de renovaci\'on, fecha vencimiento y observaciones.
    \end{itemize}
  \item Registro de veh\'iculos
    \begin{itemize}
      \item Datos del veh\'iculo: n\'umero de patente, marca, modelo.
      \item Datos relacionados al servicio: antiguedad en el servicio, capacidad de transporte y estado.
      \item Datos relacionados al estado: El estado de un veh\'iculo puede ser ``en uso'' o ``en reparaci\'on''. 
      De estar el veh\'iculo en reparaci\'on se registra la fecha de ingreso al taller.
    \end{itemize}
  \item Registro de recorridos
    \begin{itemize}
      \item Datos de origen y destino: ciudad, calle y n\'umero.
      \item Los recorridos pueden tener una o m\'as rutas posibles que unen el origen y el destino.
      \item Datos de las rutas: descripci\'on, largo, tiempo estimado del recorrido, peajes, condiciones
      del camino y condiciones clim\'aticas (seg\'un el per\'iodo del a\~no).
    \end{itemize}  
  \end{enumerate}
\item Control de viajes: el sistema debe ser capaz de ayudar a administrar los viajes que la empresa realiza y sus controles.
  \begin{enumerate}[label=\Roman{*}]
  \item Viajes planificados
    \begin{itemize}
      \item Datos a registrar: recorrido, fecha y hora de partida, fecha y hora de llegada estimada, veh\'iculo y choferes asignados
      \item Los viajes los planifica la empresa con los recursos que cuenta, como ser veh\'iculos en uso y choferes habilitados. 
      \item Cada viaje tiene un m\'aximo de tres choferes asignados.
    \end{itemize}  
  \item Viajes realizados
    \begin{itemize}
      \item Datos a registrar: fecha y hora de llegada real, ruta elegida y si hubo problemas o demoras en el camino.
      \item Un mismo recorrido podría ser realizado periódicamente por el mismo vehículo y los mismos choferes en distintas fechas.
    \end{itemize}  
  \item Registro de controles
    \begin{itemize}
      \item Datos: tipo de test (visi\'on, alcoholemia, ect.), chofer, viaje relacionado, fecha de realizacion y resultado.
      \item Los controles se hacen a choferes asingados a viajes planificados, antes de la realizaci\'on del mismo.
    \end{itemize}  
  \end{enumerate}
\item Procedimientos y pol\'iticas
  El sistema debe satistfacer las funcionalidades identificadas en la secci\'on ``Comentario de la c\'atedra''
  \begin{enumerate}[label=\Roman{*}]
  \item Recorridos para los cuales se usaron todas las rutas asociadas en viajes del a\~no pasado
  \item Recorridos con mas de una ruta asociada
  \item Estado y viajes realizados por veh\'iculo
  \item Lista de choferes que han utilizado, en el \'ultimo semestre, todos los veh\'iculos de menos de dos a\~nos de antiguedad.
  \item Implemtaci\'on de alguna restricci\'on adicional que surja del dise\~no.
  \end{enumerate}
\end{enumerate}
